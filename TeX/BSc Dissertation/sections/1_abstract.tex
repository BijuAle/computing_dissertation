\newgeometry{margin=1.5in}
\twocolumn[
	\maketitle 
	\begin{onecolabstract}
    Light from distant galaxies is a goldmine of information to determine attributes such as its redshifts and even its morphology. However, measuring the individual shift in wavelengths based on the localized spectral fingerprint of an associated element is not feasible when hundreds of billions of galaxies populate space. And in other cases, spectroscopic observations are simply unavailable for many galaxies. In this paper, the Decision Tree regression classifier is trained with the color indices at 5 bands with the corresponding photometric redshifts. The performance of the learning model is assessed with the median residual method. 10-fold cross-validation is used to improve its performance. Quasars also populated the dataset with median residual after regression at $\approx0.074$ and for galaxies at $\approx0.016$. The spectrum flux is also used in predicting the morphology of galaxies. With the decision tree, 80\% accuracy compared to true values of morphologies, was achieved. Ensemble Learning using the random forest improved the accuracy by ~6-7\%.
    
    \textbf{Keywords:} Machine Learning, Astronomy, Redshift, Morphology, Decision Tree, Random Forest, Galaxy, Ensemble Method, Sloan Digital Sky Survey (SDSS), Galaxy Zoo, Hubble Sequence, Data-driven Astronomy
	 \end{onecolabstract}
]
\newgeometry{margin=0.6in}